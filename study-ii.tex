\section{Introduction}
\blindtext

\section{Example Python code}
Below shows an example of Python code block using \texttt{Pyleoclim} \citep{pyleoclim}:
\begin{lstlisting}
import pyleoclim as pyleo

url = 'http://wiki.linked.earth/wiki/index.php/Special:WTLiPD?op=export&lipdid=MD982176.Stott.2004'
data = pyleo.Lipd(usr_path=url)
ts_list = data.to_tso()
ts_sst = pyleo.LipdSeries(ts_list)

# OUTPUT BELOW
# extracting paleoData...
# extracting: MD982176.Stott.2004
# Created time series: 6 entries
# 0 :  MD982176.Stott.2004 :  marine sediment :  depth
# 1 :  MD982176.Stott.2004 :  marine sediment :  yrbp
# 2 :  MD982176.Stott.2004 :  marine sediment :  d18og.rub
# 3 :  MD982176.Stott.2004 :  marine sediment :  d18ow-s
# 4 :  MD982176.Stott.2004 :  marine sediment :  mg/ca-g.rub
# 5 :  MD982176.Stott.2004 :  marine sediment :  sst
# 
# Enter the number of the variable you wish to use:  5
\end{lstlisting}

\section{Example citations}
Please check the ``references.bib'' file to see the details of the different types of citations.

\subsection{Journal article}
Example journal article \citet{zhu_resolving_2020}.

\subsection{Book chapter}
Example book chapter \citet{ENSObook2020:ch05}.

\subsection{Software}
Example software \citet{feng_zhu_2019_3590258}.

\section{Conclusion}
\blindtext